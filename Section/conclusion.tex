We calculated the ground-state energies of first row atoms and their corresponding ions and hydrides to an accuracy of $0.1$ mHa both with and without the adiabatic assumption. We found the ionization energies of the atoms to be independent of the adiabatic assumption, suggesting that the energy difference between the adiabatic and non-adiabatic ground states is due to the zero point motion of the nuclei at the energy scales of interest. The atomization energies of simple hydrides, however, were significantly different in the adiabatic than in the non-adiabatic limit.   We showed that it is necessary to include non-adiabatic effects to accurately predict the experimental values of atomization energies for these simple hydrides.

These calculations also verified the validity of our wave function ansatz, namely it does indeed produce a high quality electron-ion trial wavefunction from a good electron wavefunction. This technique also has the potential to solve interesting larger-scale problems due to its ease of implementation as well as the polynomial scaling in computational time with respect to the number of electrons.  This technique can be generalized quite easily to deal with larger systems.