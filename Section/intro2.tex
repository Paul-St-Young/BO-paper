Despite the presence of numerous high lever benchmarking studies for first row atoms and hydrides in the literature, the authors almost always worked within the Born-Oppenheimer approximation. While the correlation between electronic and ionic motions often only represents a tiny fraction of the total energy of a system, there are cases where non-adiabatic effects are essential in reproducing the correct physical phenomena. It has been shown that the inclusion of non-adiabatic effects is crucial for accurately characterizing the stability of atomic hydrogen as well as predicting its ground state crystal structure \cite{Ceperley_1987,Natoli_1993,Natoli_1995}. They also play an important role in distinguishing the hydrogen atom transfer (HAT) and proton-coupled electron transfer (PCET) mechanisms in benzyltoluene and phenoxyl-phenol \cite{Sirjoosingh_PCET}. Non-adiabatic effects will not be as pronounced in the highly adiabatic first row atoms and hydrides, nevertheless the energetic contribution of these effects have not been fully quantified.

Since non-adiabatic effects are often very small on the electronic energy scale, highly accurate methods are required to resolve them. To the best of our knowledge, the most accurate method capable of including non-adiabatic effects involves the use of explicitly correlated Gaussian (ECG) basis. The ECG method has been successful in producing highly accurate ground state energy, ionization energies as well as other observables. However, the current implementation of the method suffers from factorial scaling in computation time with the number of identical particles. Thus it has only been applied to small systems such as $\text{H}_2^+,\text{H}_2,\text{He}_2^+,\text{He}_2$ with the biggest system being BH. On the other hand, a much more scalable way to incorporates non-adiabatic effects is to add corrections to a basic coupled cluster calculation\cite{Feller_Corrections}. In such an approach, a frozen-core coupled cluster singles and doubles (FC-CCSD) \cite{Purvis_CCSD} calculation serves as a first approximation for the ground-state energies. Then a series of corrections including core/valance, spin-orbit coupling, higher order correlation, zero point motion, diagonal Born-Oppenheimer and scalar relativistic effects are introduced. While this method gives a detailed breakdown of the various energetic contributions, it does not capture the full effect of non-adiabaticity. In particular, only diagonal Born-Oppenheimer correction is included while the off-diagonal terms are ignored. In addition, each correction is non-trivial to calculate and extrapolation formulae are often needed, resulting in uncontrolled errors that have to be estimated either empirically or analyzed on a molecule-by-molecule basis \cite{Feller_Error}. Other promising methods include nuclear-electronic orbital (NEO) Hatree-Fock (HF) \cite{Sharon_NEO}, NEO explicitly correlated HF (NEO-ECHF) \cite{Sharon_NEOX,Sharon_NEOX1,Sharon_NEOX2}, path integral Monte Carlo \cite{Ilkka_Path,Ilkka_Path1,Ilkka_Path2} and multicomponent density functional theory \cite{Sharon_NEO-DFT,Sharon_NEO-DFT1,Sharon_NEO-DFT2,Sharon_NEO-DFT3,Gross_NEO-DFT,Gross_NEO-DFT1}. While these methods can be applied to larger systems, it would be difficult to deliver the same kind of accuracy that ECG offers without significant development. 

Quantum Monte Carlo (QMC) methods enjoy polynomial scaling with the number of particles while still being ranked among the most accurate methods in electronic structure calculations. In particular, the fixed-node diffusion Monte Carlo (FN-DMC) method alleviates the sign problem in treating Fermi statistics \cite{Ceperley_HEG,Ceperley_QMC} and is known to produce highly accurate ground state energies for atoms, molecules and solids \cite{QMC_Review}. The accuracy of the method is only limited by the quality of the nodal surface of the trial wavefunction and the finite length of computation time. With the ansatz recently developed by Tubman et. al. \cite{Tubman_ECG}, we were able to construct trial wavefunctions with very good approximation of the nodal surface including both electronic and ionic degrees of freedom.