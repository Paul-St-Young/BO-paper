The results of this work given in Tables \ref{tab:ionization} and \ref{tab:atomization} were obtained from simulations where we moved the nuclei as often as the electrons, i.e., we have used the same time-step for each particle. However, we could also use different sized time-steps for each different particle species as long as they are, e.g., multiples of the smallest time-step. Due to the Trotter expansion \cite{lester1} this only affects the diffusion term in the diffusion Monte Carlo method, not the branching part. Thus, while applying the kinetic projection operator of the electrons for $M$ times with time-step $\tau$, a heavier particle can use a kinetic projector with time-step $\beta=M\tau$, which is applied only once. Our numerical tests with LiH molecule show that even with $M=1000$ for the proton, and $M=4000$ for the Li nucleus the results coincide with those shown in Table II, in which that nuclei were moved as often as the electrons. However, we believe that for the Li nucleus $M$ could be larger, since in the diffusion term the exponential has a prefactor of $m/2\tau$. Therefore, it is possible that $M$ could be roughly equal to the mass of the particle, since then the prefactor $m/2M\tau$ would roughly be equal to the one found accurate for the electrons.

The major benefit of this ''multi-time-stepping'' procedure is that going beyond the dragged-node approximation one needs to calculate on-the-fly updates for the electronic wave function each time the nuclei are moved. Thus, the possibility to make external program calls only every thousand steps in case of nonadiabatic hydrogen, and every four thousand (or even ten thousand) steps in case of Li nuclei will enable even more accurate calculations for more complex systems.