Under the Born-Oppenheimer (BO) approximation \cite{BO}, the full molecular wavefunction is considered to be a product of an electron component and an ion component. The electron problem is solved by "clamping" the ions to their equilibrium positions. Then the ion problem is solved using the equilibrium electron distribution as an effective potential energy surface. This assumption is based on the fact that protons are roughly 1836 times as heavy as electrons. Therefore ions, being even heavier than protons, move much more slowly than electrons. This allows the electrons to relax instantaneously and adiabatically to their equilibrium states as the ions move around. The BO approximation is excellent in most cases, but recent advances suggest one might be missing important physics by adopting this approximation in certain cases, particularly when a conical intersection is present \cite{Cederbaum_BO}. For instance, the prediction of ground state structure of atomic hydrogen \cite{Ceperley_1987,Natoli_1993,Natoli_1995}, the dissociation of hydrogen at finite temperature \cite{Mazzola_FiniteT} and photon coupled electron transfer (PCET) in phenoxyl-phenol \cite{Sirjoosingh_PCET} were only possible due to the inclusion of non-adiabatic effects. We acknowledge the subtle difference between the Born-Oppenheimer approximation and the adiabatic assumption pointed out by Worth and Cederbaum \cite{Cederbaum_BO}, namely the former includes the kinetic energy of nuclei on a single electronic potential surface whereas the latter doesn't. However, we will use the phrases non-Born-Oppenheimer and non-adiabatic interchangeably since they both mean using the exact molecular Hamiltonian in the Quantum Monte Carlo (QMC) community.

While there have been many cases in the literature where non-adiabatic effects were included for atomic and molecular systems, they tend to require a great deal of computational as well as human effort. One approach is to add corrections to a basic coupled cluster theory to produce highly accurate ground state energies for atoms and molecules\cite{Feller_Corrections}. In such an approach, a frozen-core coupled cluster singles and doubles (FC-CCSD) \cite{Purvis_CCSD} calculation serves as a first approximation for the ground-state energies. Then a series of corrections including core/valance, spin-orbit coupling, higher order correlation, zero point motion, diagonal Born-Oppenheimer and scalar relativistic effects are introduced. Each correction involves a non-trivial and often costly calculation. In addition, extrapolation formulae are sometimes needed to take the result to the complete basis set limit. This method produces highly accurate expectation values, but contains uncontrolled errors that have to be estimated either empirically or analyzed on a molecule-by-molecule basis \cite{Feller_Error}. The addition of uncertainties in each correction also contributes to a rather large error bar, making it difficult to obtain a small confidence interval. Another approach in a more recent development is to use the explicitly correlated Gaussian (ECG) basis \cite{Adamowicz_ECG,Mitroy_ECG}. This approach is capable of calculating the ground-state energies of small molecules to an incredibly high accuracy. Unfortunately, the current implementation of ECG cannot be applied to moderately-sized systems due to factorial scaling in computational cost with the number of identical particles \cite{Bubin_BH_noBO}. Other methods with less aggressive scaling include nuclear-electronic orbital (NEO) Hatree-Fock (HF) \cite{Sharon_NEO}, NEO explicitly correlated HF (NEO-ECHF) \cite{Sharon_NEOX,Sharon_NEOX1,Sharon_NEOX2}, path integral Monte Carlo \cite{Ilkka_Path,Ilkka_Path1,Ilkka_Path2} and multicomponent density functional theory \cite{Sharon_NEO-DFT,Sharon_NEO-DFT1,Sharon_NEO-DFT2,Sharon_NEO-DFT3,Gross_NEO-DFT,Gross_NEO-DFT1}. While these methods can be applied to larger systems, it would be difficult to deliver the same kind of accuracy that ECG offers without significant development. In this paper, we would like to demonstrate that, with simple modifications \cite{Tubman_ECG}, the fixed-node diffusion Monte Carlo (FN-DMC) algorithm is capable of producing results on par with and even better than the most sophisticated quantum chemistry methods for systems as small as BH.