The ionization energies are listed in Table \ref{tab:ionization} and they agree well with experimental results. Notice that even though ground state energies change significantly with the inclusion of non-adiabatic effects, the ionization energies match with or without the adiabatic assumption. This suggests that for atomic systems, coupling between valence electron and ion motions is small. The difference in ground state energies can be entirely attributed to the zero point motion of the nuclei. Physically, this means that for all first row atoms, the outer most electron is screened from the nucleus and all of the energy required for its removal can be attributed to its interaction with the rest of the electrons in the atom.

For the LiH molecule we are also interested in calculating the electron affinity for comparison to ECG results. We calculated the ground state energy of LiH$^-$ to be $-8.08220(2)$~Ha for the case of clamped nuclei.  With non-adiabatic effects included our result is  $-8.07811(3)$~Ha. Our non-adiabatic result is in good agreement with a previous ECG study \cite{Bubin_LiH_noBO} which reported a value of $-8.07856887$~Ha. We report an electron affinity of $0.01191(4)$~Ha which is can be compared to the ECG prediction of $0.012132(2)$~Ha and agrees with experiment, $0.0126(4)$~Ha.  We note that $\text{LiH}$ ground state energies which we compare against are mislabeled in Ref.\cite{Bubin_LiH_noBO}, with $\text{LiH}^-$ and LiD being switched.