There has been several recent discoveries that suggest that quantum wave functions, which include both electronic and ionic degrees of freedom, have many interesting properties that have yet to be explored.  This includes the development of equations that exactly factorize a wave function into electron and ionic components~\cite{cederbaum1}, the disappearance of conical intersections in wave functions of model systems~\cite{gross2014}, and the use of quantum entanglement to study electronic and ionic density matrices~\cite{boent}.  Extending such studies to realistic systems is of broad interest and will considerably expand our understanding of electron-ion systems.   However, treatment of \textit{ab initio} electron-ion systems is challenging and applications have thus been limited.   The most accurate simulations of electron-ion wave functions are generally done with very specialized wave functions, which are limited to rather small systems sizes \cite{mitroy2013}.  

As a framework to address these problems in general realistic systems, we recently demonstrated that quantum Monte Carlo (QMC) can be combined with quantum chemistry techniques to generate electron-ion wave functions~\cite{Tubman_ECG}.  We treated realistic molecular systems and demonstrated that our method can be scaled to larger systems than previously considered while maintaing a highly accurate wave function.  In the following we extend our previous work by considering the simulation of a larger benchmark set of atoms and molecules.  We calculate ionization energies and dissociation energies which can be directly compared with previous benchmarking results.  We also consider using multiple time steps for different species in the imaginary time propagator, which is also developed and tested.  Additionally we consider other estimators and the errors associated by using a mixed estimator.