Ground state energies are calculated for first row atoms and ions and hydrides with and without the adiabatic assumption, see Table \ref{tab:ionization} and \ref{tab:atomization}. The ground state geometries for LiH,BeH and BH are chosen to be the ECG-optimized distances for a fair comparison with the ECG method and the geometries for the rest of the hydrides are taken from experimental data \cite{CCCBDB}. We first perform a CAS(m,n) calculation (m electrons into n active orbitals), then the MCSCF optimized orbitals are used in a SOCI calculation that includes single and double excitations of the m electrons into all of the remaining valance orbitals. We include all CSFs with coefficients bigger than some cutoff $\epsilon$ to lend reasonable flexibility to the wavefunction during optimization. We include as many CSFs as possible to maximize the flexibility of the wavefunction. However, the inclusion of too many CSFs with small expansion coefficients can introduces noise as they requires a large number of samples in the optimization step to be optimized. We have chosen to restrict the number of CSFs in the wave function to be $\sim$1000 in all systems. Optimization was performed with roughly $10^7$ statistically independent samples and we chose a cost function consisting of equal parts average local energy and reweighted variance. We performed timestep extrapolation for all of the tested systems. Five timesteps from $0.005~\text{Ha}^{-1}$ to $0.001~\text{Ha}^{-1}$ were used for all systems in the adiabatic FN-DMC.

The clamped nuclei ground state FN-DMC energies  are consistently equal across all systems, within error bars, with a recent QMC benchmark study~\cite{Seth_Bench}. This is an interesting coincidence since we used a different approach in optimizing our wave functions.   In particular our large multi-determinant expansions, can be compared with the approach used by Seth {\it et al.}~\cite{Seth_Bench} which used moderately-sized multi-determinant expansions ($\sim$ 100 CSF) with a backflow transformation. 1
 
 For the atomic systems, there are two ECG calculations of non-adiabatic ground state energies we can use as benchmark.  The non-adiabatic ground-state energies for Be and B ($-14.66643(2)$~Ha and $-24.65244(3)$~Ha) are in agreement with ECG results  to an accuracy of less than 0.2 mHa (-$14.66643544$~Ha \cite{Bubin_BeH_noBO} and -$24.652598$~Ha~\cite{Bubin_BH_noBO}).

We performed a study over diatomic systems, the results of which are presented in Table \ref{tab:atomization}. For these diatomic systems, there are Feller benchmark results~\cite{Feller_Corrections}, for which we can compare our results againsts. 

To make the comparison against the semi-empirical results, we took the reference energies from the last column of Table VI of Ref.~\cite{Feller_Corrections} and subtracted the corrections in the $\Delta E_{SR}$ and SO columns for comparison with our non-adiabatic energies.  For the comparison with our adiabatic energies we subtracted the DBOC and ZPE corrections.  Corrections from spin-orbit coupling and relativistic effects are not used, as they are not included in our Hamiltonian.