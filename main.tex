\documentclass[pra,superscriptaddress,groupedaddress,twocolumn]{revtex4}
\usepackage{graphicx}  % needed for figures
\usepackage{siunitx}   % for decimal alignment in tables
\usepackage{bm,amssymb,amsmath}    % for math
\usepackage[bookmarks]{hyperref}
\usepackage{threeparttable}

\begin{document}

\title{Accurate predictions of ionization and atomization energies without the Born-Oppenheimer approximation}
\input{Section/authors}
\begin{abstract}
In this work we calculate the non-relativistic ground-state energies of atomic and molecular systems both with and without the Born Oppenheimer approximation. For this we utilize the fixed-node diffusion Monte Carlo method, in which the nodes depend both on the electronic and ionic positions. We report ground state energies, ionization energies and atomization energies to an accuracy of less than $1$ mHa for all but the largest systems. We find the ionization energies of the atoms to be independent of the adiabatic assumption except for O and F where the non-adiabatic and adiabatic ionization energies differ by less than 0.5 mHa. This suggests that the coupling between the nucleus and valence electrons is not important in the ionization process. The atomization energies, however, are influenced by the non-adiabatic coupling of electrons and nuclei at the mHa level. We demonstrate that the fixed node approximation provides a highly accurate and scalable approach to treating molecule systems beyond the Born Oppenheimer approximation.
\end{abstract}
\maketitle

\section{Introduction}
There have been several recent discoveries~\cite{cederbaum1,gross2014,boent} that suggest that quantum wave functions, which include both electronic and ionic degrees of freedom, have many interesting properties that have yet to be explored.  This includes the development of equations that exactly factorize a wave function into electron and ionic components~\cite{cederbaum1}, the disappearance of conical intersections in wave functions of model systems~\cite{gross2014}, and the use of quantum entanglement to study electronic and ionic density matrices~\cite{boent}.  Extending such studies to realistic systems is of broad interest and will considerably expand our understanding of electron-ion systems. However, treatment of \textit{ab initio} electron-ion systems is challenging and applications have thus been limited.   The most accurate simulations of electron-ion wave functions are generally done with very specialized wave functions, which are limited to rather small systems \cite{mitroy2013}.  

As a framework to address these problems in general realistic systems, we recently demonstrated that quantum Monte Carlo (QMC) can be combined with quantum chemistry techniques to generate electron-ion wave functions~\cite{Tubman_ECG}.  We treated realistic molecular systems and demonstrated that our method can be scaled to larger systems than previously considered while maintaining a highly accurate wave function. In the following we extend our previous work by considering the simulation of a larger benchmark set of atoms and molecules.  We calculate ionization energies and atomization energies which can be directly compared with previous benchmarking results.

\section{Method}
\subsection{Fixed-Node Diffusion Monte Carlo (FN-DMC)}
Diffusion Monte Carlo is a projector method that evolves a trial wavefunction in imaginary time and projects out the ground-state wavefunction.  For practical simulations of fermions, the fixed-node approximation is introduced, which depends only on the set of electronic positions where a trial wave functions is equal to zero.  This approximation is different than approximations typically used in quantum chemistry calculations, and in this work we demonstrate that we can generate high quality nodal surfaces for a range of systems which include full electron-ion wave functions. 

If the trial wavefunction has the same nodal surface as the exact ground-state wavefunction, FN-DMC will obtain the exact ground-state energy.  Approximate nodal surfaces can be generated through optimization of the full wave function. Such approximate nodal surfaces have been tested and validated on a wide range of systems, and consistently provide an excellent approximation of the exact ground-state energy,  comparable to the state of the art in \textit{ab initio} simulations~\cite{grossman1}. In addition the energies generated with FN-DMC are variational with regards to the ground state energy.

In all but a handful of previous QMC simulations, ions are "clamped" to their equilibrium positions. Recent advances~\cite{Nightingale_Linear,Umrigar_Linear,Brown_Bench} have made it possible to optimize thousands of wave function parameters simultaneously with variational Monte Carlo within the clamped nuclei approximation. However, Such an assumption is not fundamentally required by FN-DMC.  In our previous work we found that the most important effect to optimize for were the nodes due to electron-electron correlations~\cite{Tubman_ECG}, and in this regard we use more sophisticated electronic terms in the wave function than the ion part of the wave function.

\subsection{Electron Wavefunction and Optimization}
There are several different approaches for generating high quality wave functions ~\cite{Umrigar_Alleviation,Toulouse_Bench, Brown_Bench,Seth_Bench}. We use an initial guess for the wavefunction that is generated from complete active space self-consistent field (CASSCF) \cite{Chaban_MCSCF,Szabo} calculation using the quantum chemistry package GAMESS \cite{GAMESS}. The optimized orbitals are then used in a singles and doubles configuration interaction (CISD) calculation to generate a series of configuration state functions (CSF)~\cite{Clark_Bench}. The multi-CSF expansion of the wavefunction can be expressed in the following form,
\begin{align}
\Psi_{\text{CISD}}(\vec{r};\vec{R}_e)=\sum\limits_{i=1}^{N_{\text{CSF}}}\alpha_i\phi_i(\vec{r};\vec{R}_e), \label{eq:psi_gms}
\end{align}
where $\vec{r}$ refers to the spatial coordinates of all the electrons, $\vec{R}_e$ refers to the equilibrium positions of all the ions. $\phi_i(\vec{r})$ and $\vec{\alpha}=\{\alpha_1,\alpha_2,\dots\}$ are the CSF and CI coefficients generated from CISD. We used the cc-pV5Z basis for all the atomic systems and Roos Augmented Triple Zeta ANO basis for molecular systems~\cite{dunning,roos}. We then impose the cusp condition on each molecular orbital~\cite{cusp} and add a Jastrow factor to the wave function to include electron correlation~\cite{Kato}. Our Jastrow factor contains one, two and three body terms. The full electronic wave function being optimized is then
\begin{align}
\psi_e(\vec{r})=e^{J(\vec{r},\vec{\beta})}\Psi_{\text{CISD}}(\vec{r};\vec{R}_e)\label{eq:psie}
\end{align}
We optimized the CSF and Jastrow coefficients $\vec{\alpha},\vec{\beta}$ simultaneously with QMCPACK \cite{QMCPACK}. Optimization is performed with the ions clamped to their equilibrium positions. We include all CSFs with coefficients bigger than some cutoff $\epsilon$ to lend reasonable flexibility to the wavefunction during optimization. We include as many CSFs as possible to maximize the flexibility of the wavefunction. However, the inclusion of too many CSFs with small expansion coefficients can introduces noise as they requires a large number of samples in the optimization step to be optimized. We have chosen $\epsilon$ to restrict the number of CSFs in the wave function to be $\sim$1000 in all systems. Optimization was performed with the linear method with roughly $10^6$ statistically independent samples and we chose a cost function consisting of equal parts average local energy and reweighted variance.

\subsection{Electron-Ion Wavefunction}

\begin{figure}[t]
\includegraphics[width=9cm]{fig1a.pdf}
\includegraphics[width=9cm]{fig1b.pdf}
\caption{ Symmetries for simulation of atomic and molecule systems in QMC {\bf (a)} For atomic systems we can consider the entire wavefunction shifting with the ion. This process can be visualized by following a contour of the wavefunction. The thick dashed circle represents a contour of the electron wavefunction when the proton is at its reference position $\vec{R}_o$ and the thin dashed circle represents the same contour when the proton has moved to a new position $\vec{R}$. To evaluate the ion-dependent electron wavefunction $\bar{\psi}_e(\vec{r},\vec{R})$, we simply map the electron to its proper place in the reference wavefunction $\psi_e(\vec{r})$. That is, $\bar{\psi}_e(\vec{r},\vec{R})=\bar{\psi}_e(\vec{r}+\vec{s},\vec{R}_o)=\psi_e(\vec{r}+\vec{s})$ where $\vec{s}$ is the shift required to put the proton back to its reference position. {\bf (b)} For H$_2^+$, we pick one of the protons as an "anchor" and approximate the new wavefunction by dragging the reference wavefunction with the "anchor" proton. We also rotate the wavefunction to align its axis of symmetry with the orientation of the two protons. \label{fig:drag}}
\end{figure}

Once a satisfactory electronic wave function has been obtained, we construct the electron-ion wave function using the ansatz we previously investigated~\cite{Tubman_ECG},
\begin{align}
\Psi_{ei}(\vec{r},\vec{R})=\psi_I(\vec{R})\bar{\psi}_e(\vec{r},\vec{R}), \label{eq:psi}
\end{align}
where $\vec{R}$ includes spatial coordinates of all ions. The ionic wave function consists of simple products of Gaussian wave functions over each nuclei pair,
\begin{align}
\psi_I(\vec{R})\propto \prod\limits_{i,j>i}e^{-a(\vert \vec{R}_i-\vec{R}_j\vert-b_{ij})^2},
\label{wfs_ions}
\end{align}
where $a$ is a coefficient for the ion wave function that we optimize and $b_{ij}$ are taken to be the equilibrium distances between the nuclei. The equilibrium geometries for LiH, BeH and BH are chosen to be the ECG-optimized distances for a comparison with the ECG method and the geometries for the rest of the hydrides are taken from experimental data \cite{CCCBDB}. FN-DMC was then performed with the fully optimized electron-ion wave function. We performed timestep extrapolation for all of the tested systems. At least four timesteps from $0.005~\text{Ha}^{-1}$ to $0.001~\text{Ha}^{-1}$ were used for all systems in the adiabatic FN-DMC and at least three timesteps from $0.001~\text{Ha}^{-1}$ to $0.0001~\text{Ha}^{-1}$ were used in non-adiabatic FN-DMC.

Due to the locality of the Gaussian basis set used in the quantum chemistry calculations for constructing the electronic wave function, the nodes change based on the ionic positions, which we have previously called the dragged-node approximation. In particular, the nodes only coincide with those obtained from the quantum chemistry calculation when the ions are at their equilibrium positions $\bar{\psi}_e(\vec{r},\vec{R}_e)=\psi_e(\vec{r})$. Although there are approaches for going beyond this approximation, it was demonstrated to be highly accurate over a range of molecules in previous work.~\cite{Tubman_ECG} For the systems considered here, we can impose various symmetries of the Hamiltonian onto the wave function that arise from the relative motion of the ions. In Fig.~\ref{fig:drag} we demonstrate this strategy for the simple cases of a hydrogen atom and a H$_2^+$ molecular ion. Although the dragged-node technique is developed with atomic and diatomic systems in mind, it is not difficult to generalize it for use in larger systems or even apply to parts of a bigger system, e.g., treating light ions as quantum particles and heavy ions as "clamped". 

\section{Results and Discussion}
\begin{table*}[t!]
\setlength{\extrarowheight}{1pt}
\begin{threeparttable}

\caption{Ground-state energies for atoms and ions and the ionization energies for the atoms:  fixed-node DMC results of this work (FN-DMC) for atoms and ions with and without the Born-Oppenheimer Approximation. The rows marked with bolded \textbf{FN-DMC} are our non-adiabatic results. The ionization potentials (IP) are reported in the last section of the table. Energies are given in units of Hartree. For the highly accurate Hylleraas and ECG results, up to 8 digits are reported in the table. \label{tab:ionization}}
\begin{tabular}
% siunitx setup
{
 l
 S[table-format=1.6]
 S[table-format=5.6]
 S[table-format=5.6]
 S[table-format=4.6]
 S[table-format=4.6]
 S[table-format=4.6]
 S[table-format=4.6]
 S[table-format=4.6]
}

\hline\hline
\multicolumn{1}{c}{Atom} & 
\multicolumn{1}{c}{Li$(^2$S)} &
\multicolumn{1}{c}{Be$(^1$S)} &
\multicolumn{1}{c}{B$(^2$P)} &
\multicolumn{1}{c}{C$(^3$P)} &
\multicolumn{1}{c}{N$(^4$S)} &
\multicolumn{1}{c}{O$(^3$P)} &
\multicolumn{1}{c}{F$(^2$P)} \\ 
\hline
\multicolumn{1}{c}{} & 
\multicolumn{1}{c}{} &
\multicolumn{1}{c}{} &
\multicolumn{1}{c}{clamped-ion} &
\multicolumn{1}{c}{} &
\multicolumn{1}{c}{} &
\multicolumn{1}{c}{} &
\multicolumn{1}{c}{} \\
FN-DMC & \text{-}7.478056(4) & \text{-}14.66732(1) & \text{-}24.65377(1) & \text{-}37.84449(2) & \text{-}54.58858(3) & \text{-}75.06576(4) & \text{-}99.7316(1) \\
Seth DMC \cite{Seth_Bench} & \text{-}7.478067(5) & \text{-}14.667306(7) & \text{-}24.65379(3) & \text{-}37.84446(6) & \text{-}54.58867(8) & \text{-}75.0654(1) & \text{-}99.7318(1) \\
$E_{\text{ref}}$\tnote{a} &  \text{-}7.478060 & \text{-}14.667356 & \text{-}24.653866 & \text{-}37.8450 & \text{-}54.5892 & \text{-}75.0673 & \text{-}99.7339 \\
\multicolumn{1}{c}{} & 
\multicolumn{1}{c}{} &
\multicolumn{1}{c}{} &
\multicolumn{1}{c}{non-adiabatic} &
\multicolumn{1}{c}{} &
\multicolumn{1}{c}{} &
\multicolumn{1}{c}{} &
\multicolumn{1}{c}{} \\
\textbf{FN-DMC} & \text{-}7.47743(1) & \text{-}14.66643(2) & \text{-}24.65240(5) & \text{-}37.8428(1) & \text{-}54.58639(8) & \text{-}75.06298(9) & \text{-}99.72922(7) \\
ECG \tnote{b} & \text{-}7.477452 & \text{-}14.666435 & \text{-}24.652624 & \text{-}37.841621 & N/A & N/A & N/A \\
\hline

\multicolumn{1}{c}{Ion} & 
\multicolumn{1}{c}{$\text{Li}^+(^1$S)} &
\multicolumn{1}{c}{$\text{Be}^+(^2$S)} &
\multicolumn{1}{c}{$\text{B}^+(^1$S)} &
\multicolumn{1}{c}{$\text{C}^+(^2$P)} &
\multicolumn{1}{c}{$\text{N}^+(^4$S)} &
\multicolumn{1}{c}{$\text{O}^+(^3$P)} &
\multicolumn{1}{c}{$\text{F}^+(^2$P)} \\ 
\hline
\multicolumn{1}{c}{} & 
\multicolumn{1}{c}{} &
\multicolumn{1}{c}{} &
\multicolumn{1}{c}{clamped-ion} &
\multicolumn{1}{c}{} &
\multicolumn{1}{c}{} &
\multicolumn{1}{c}{} &
\multicolumn{1}{c}{} \\
FN-DMC & \text{-}7.279919(4) & \text{-}14.324763(6) & \text{-}24.34884(1) & \text{-}37.43075(2) & \text{-}54.05376(3) & \text{-}74.56588(4) & \text{-}99.0913(1) \\
Seth DMC \cite{Seth_Bench} & \text{-}7.279914(3) & \text{-}14.324761(3) & \text{-}24.34887(2) & \text{-}37.43073(4) & \text{-}54.05383(7) & \text{-}74.56662(7) & \text{-}99.0911(2) \\
$E_{\text{ref}}$\tnote{c} & \text{-}7.279913 & \text{-}14.324763 & \text{-}24.348884 & \text{-}37.430880 & \text{-}54.0546 & \text{-}74.5668 & \text{-}99.0928 \\ 
\multicolumn{1}{c}{} &
\multicolumn{1}{c}{} &
\multicolumn{1}{c}{} &
\multicolumn{1}{c}{non-adiabatic} &
\multicolumn{1}{c}{} &
\multicolumn{1}{c}{} &
\multicolumn{1}{c}{} &
\multicolumn{1}{c}{} \\
\textbf{FN-DMC} & \text{-}7.27931(4) & \text{-}14.32386(2) & \text{-}24.34748(3) & \text{-}37.42901(4) & \text{-}54.05164(2) & \text{-}74.56336(5) & \text{-}99.08843(5) \\
ECG \tnote{d} & N/A &  \text{-}14.323863 &  \text{-}24.347641 &  \text{-}37.429169 & N/A & N/A & N/A \\
\hline
\multicolumn{1}{c}{} & 
\multicolumn{1}{c}{} &
\multicolumn{1}{c}{} &
\multicolumn{1}{c}{clamped-ion} &
\multicolumn{1}{c}{} &
\multicolumn{1}{c}{} &
\multicolumn{1}{c}{} &
\multicolumn{1}{c}{} \\
IP (FN-DMC) & 0.19814(1) & 0.34257(1) & 0.30493(1) & 0.41374(3) & 0.53482(4) & 0.49988(5) & 0.6403(1) \\
\multicolumn{1}{c}{} & 
\multicolumn{1}{c}{} &
\multicolumn{1}{c}{} &
\multicolumn{1}{c}{non-adiabatic} &
\multicolumn{1}{c}{} &
\multicolumn{1}{c}{} &
\multicolumn{1}{c}{} &
\multicolumn{1}{c}{} \\
IP (FN-DMC) & 0.19812(4) & 0.34257(3) & 0.30492(6) & 0.41375(7) & 0.53475(8) & 0.4996(1) & 0.64079(8) \\
IP (Ref.)\tnote{e} & 0.198130 & 0.342572 & 0.304980 & 0.414014 & 0.534775 & 0.500452 & 0.640946 \\
\hline\hline
\end{tabular}

\begin{tablenotes}
\item[a] For the atomic references, we use Hylleraas result for Li~\cite{Wang_Li}, and ECG results for Be~\cite{Stanke_Be} and B~\cite{Bubin_B}. Ref.~\cite{Davidson_Atoms} is used for C,N,O,F where the ground-state energies are taken from Table XI.
\item[b] We use non-adiabatic ECG results as reference for Li~\cite{Stanke_Li0}, Be~\cite{Bubin_BeH_noBO} and B~\cite{Bubin_B}, which are converged to the true ground-state to well within 0.1 mHa. The result for C~\cite{Bubin_C}, however, may have error on the order of 1 mHa.
\item[c] For the ionic references, we use ICI result for $\text{Li}^+$~\cite{Nakashima_Li+}, Hylleraas result for $\text{Be}^+$~\cite{Puchalski_Be+} and ECG results for $\text{B}^+$~\cite{Bubin_B+} and $\text{C}^+$~\cite{Bubin_C+,mitroy2013}. Ref.~\cite{Davidson_Atoms} is used for $\text{N}^+$,$\text{O}^+$,$\text{F}^+$.
\item[d] ECG references only exist for $\text{Be}^+$~\cite{Bubin_BeH_noBO}, $\text{B}^+$~\cite{Bubin_B+} and $\text{C}^+$~\cite{Bubin_C+}
\item[e] Spin-orbit coupling and relativistic corrections~\cite{Klopper_IP} are removed from experimental data~\cite{NIST_Atoms} for comparison.
\end{tablenotes}

\end{threeparttable}
\end{table*}

\begin{figure}
\centering
\includegraphics[scale=.4]{Figures/atom-ECG}
\caption{FN-DMC ground state energies for $\text{Be}^+$,Be,$\text{B}^+$,B,$\text{C}^+$ relative to ECG references.~\cite{Stanke_Be,Puchalski_Be+,Bubin_BeH_noBO,Bubin_B,Bubin_B+,Bubin_C+} These differences provide an estimate for the fixed node error. \label{fig:atom-ECG}}
\end{figure}

\begin{figure}
\includegraphics[scale=.37]{Figures/atom-nad-ad}
\caption{Non-adiabatic contribution to ground state energies of atoms and ions. $E_n^{DMC}$ denotes the non-adiabatic ground state energy and $E_c^{DMC}$ the clamped-ion ground state energy.}
\end{figure}

\subsection{Ground State Energies}

Ground state energies are calculated for first row atoms and ions and hydrides with and without the adiabatic assumption, see Table \ref{tab:ionization} and \ref{tab:atomization}. For the small systems, Li, Be, B, Be$^+$, B$^+$, C$^+$, LiH, BeH and BH the ECG/Hylleraas results are converged to several digits beyond what can be done with normal quantum chemistry techniques. Thus these results are likely much more accurate than experimental results, and are converged to an accuracy beyond our FN-DMC results.

As an illustration of the high quality QMC techniques used in this work, we compare our atomic results with a recent QMC benchmark study~\cite{Seth_Bench}. The clamped nuclei ground state FN-DMC energies consistently agree across all systems (except for O$^{+}$), within error bars. This is an interesting coincidence since we used a different approach in optimizing our wave functions. In particular our large multi-determinant expansions, can be compared with the approach used by Seth {\it et al.}~\cite{Seth_Bench} which used moderately-sized multi-determinant expansions ($\sim$ 100 CSF) with a backflow transformation.   
 
For the atomic systems, there are five ECG calculations of non-adiabatic ground state energies we can use as benchmark. The non-adiabatic ground-state energies for Be, $\text{Be}^+$, B,$\text{B}^+$ and $\text{C}^+$ are in agreement with ECG results within 0.2 mHa as shown in Figure \ref{fig:atom-ECG} and Table \ref{tab:ionization}. We performed a study over diatomic systems, the results of which are presented in Table \ref{tab:atomization}. For these diatomic systems, there are high quality thermochemistry benchmark results for which we can compare ~\cite{Feller_Corrections}. 

To make the comparison against the thermochemistry benchmark results, we take the reference energies from the last column of Table VI of Ref.~\cite{Feller_Corrections} and subtracted the corrections in the $\Delta E_{SR}$ and SO columns for comparison with our non-adiabatic energies.  For the comparison with our adiabatic energies we subtracted the DBOC and ZPE corrections.  Corrections from spin-orbit coupling and relativistic effects are not used, as they are not included in our Hamiltonian.

\begin{figure}[h]
\centering
\includegraphics[scale=.4]{Figures/dia-ECG}
\caption{Ground state energies of LiH, BeH and BH calculated with non-adiabatic FN-DMC relative to ECG results.~\cite{Adamowicz_LiH,Koput_BeH,Miliordos_BH}.}
\end{figure}

\begin{figure}[h]
\includegraphics[scale=.4]{Figures/hydride-ZPE}
\caption{Non-adiabatic contribution to ground state energies in hydrides. The adiabatic reference energies are calculated by adding zero-point contributions from \cite{Feller_Corrections} to our clamped-nuclei results.}
\end{figure}

\subsection{Ionization Energies}

The ionization energies are listed in Table \ref{tab:ionization}.  Notice that even though ground state energies change significantly with the inclusion of non-adiabatic effects, the ionization energies are statistically indistinguishable except for O and F where the energy differences between O,O+ and F,F+ are 0.26(11)mHa and 0.49(12)mHa respectively. This suggests that for atomic systems, coupling between valence electron and ion motions is small. The difference in ground state energies can be almost entirely attributed to the zero point motion of the nuclei. Physically, this means that for all first row atoms, the outer most electron is screened from the nucleus and all of the energy required for its removal can be attributed to its interaction with the rest of the electrons in the atom.

For the LiH molecule we are also interested in calculating the electron affinity for comparison to ECG results. We calculated the ground state energy of LiH$^-$ to be $-8.08220(2)$~Ha for the case of clamped nuclei.  With non-adiabatic effects included our result is  $-8.07811(3)$~Ha. Our non-adiabatic result is in good agreement with a previous ECG study \cite{Bubin_LiH_noBO} which reported a value of $-8.07856887$~Ha. We report an electron affinity of $0.01191(4)$~Ha which is can be compared to the ECG prediction of $0.012132(2)$~Ha and agrees with experiment, $0.0126(4)$~Ha. \footnote{We note that LiH ground state energies which we compare against are mislabeled in Ref.\cite{Bubin_LiH_noBO}, with $\text{LiH}^-$ and LiD being switched.}

\begin{figure}
\centering
\includegraphics[scale=.4]{Figures/ionization}
\caption{Calculated ionization energies relative to experimental data. The calculated energies are all within 1 mHa of experiment.}
\end{figure}

\begin{table*}[t!]
\setlength{\extrarowheight}{1pt}
\begin{threeparttable}
\caption{Ground state energies and atomization energies: Fixed-Node DMC results of this work for all first row hydrides with and without the adiabatic assumption. Energies are given in units of Hartree. \label{tab:atomization}}
\begin{tabular}
% siunitx setup
{
 l
 S[table-format=1.6]
 S[table-format=6.6]
 S[table-format=5.6]
 S[table-format=4.6]
 S[table-format=4.6]
 S[table-format=4.6]
 S[table-format=4.6]
}

\hline\hline
\multicolumn{1}{c}{Molecule} & 
\multicolumn{1}{c}{LiH$(^1\Sigma^+)$} &
\multicolumn{1}{c}{BeH$(^2\Sigma^+)$} &
\multicolumn{1}{c}{BH$(^1\Sigma^+)$} &
\multicolumn{1}{c}{CH$(^2\Pi)$} &
\multicolumn{1}{c}{OH$(^2\Pi)$} &
\multicolumn{1}{c}{HF$(^1\Sigma^+)$} \\ 
\hline
\multicolumn{1}{c}{} & 
\multicolumn{1}{c}{} &
\multicolumn{1}{c}{} &
\multicolumn{1}{c}{clamped nuclei} &
\multicolumn{1}{c}{} &
\multicolumn{1}{c}{} &
\multicolumn{1}{c}{} \\
E (this work) & \text{-}8.070521(7) & \text{-}15.24793(1) & \text{-}25.28868(2) & \text{-}38.4781(1) & \text{-}75.7352(1) & \text{-}100.4556(2) \\
$E_{\text{ref}}$ \tnote{a} \cite{Adamowicz_LiH,Koput_BeH,Miliordos_BH,Davidson_Atoms,Feller_Corrections} & \text{-}8.07045 & \text{-}15.247846 & \text{-}25.287650 & \text{-}38.4792(2) & \text{-}75.7382(2) & \text{-}100.4600(3) \\
\multicolumn{1}{c}{} & 
\multicolumn{1}{c}{} &
\multicolumn{1}{c}{} &
\multicolumn{1}{c}{non-adiabatic} &
\multicolumn{1}{c}{} &
\multicolumn{1}{c}{} &
\multicolumn{1}{c}{} \\
E (this work) & \text{-}8.06620(2) & \text{-}15.24196(7) & \text{-}25.28103(4) & \text{-}38.4704(4) & \text{-}75.7237(4) & \text{-}100.4445(5) \\
ECG \cite{Bubin_LiH_noBO,Bubin_BeH_noBO,Bubin_BH_noBO} & \text{-}8.0664371(15) & \text{-}15.24203(10) & \text{-}25.2803(10) & N/A & N/A & N/A \\
\hline

\multicolumn{1}{c}{} & 
\multicolumn{1}{c}{} &
\multicolumn{1}{c}{} &
\multicolumn{1}{c}{clamped nuclei} &
\multicolumn{1}{c}{} &
\multicolumn{1}{c}{} &
\multicolumn{1}{c}{} \\
$D_{e}$ (this work) & 0.092465(8) & 0.08062(1) & 0.13491(2) & 0.13361(2) & 0.16944(4) & 0.2240(2) \\
Feller \tnote{b} \cite{Feller_Corrections} & 0.09262(5) & 0.0809(4) & 0.1354(2) & 0.1342(2) & 0.1709(2) & 0.2258(3) \\
\multicolumn{1}{c}{} & 
\multicolumn{1}{c}{} &
\multicolumn{1}{c}{} &
\multicolumn{1}{c}{non-adiabatic} &
\multicolumn{1}{c}{} &
\multicolumn{1}{c}{} &
\multicolumn{1}{c}{} \\
$D_{0}^0$ (this work) & 0.08905(2)  & 0.07580(7)  & 0.12886(5) & 0.1279(7) & 0.1610(4) & 0.2158(6) \\
Feller \tnote{c} \cite{Feller_Corrections} & 0.08940(5) & 0.0761(4) & 0.1299(2) & 0.1276(2) & 0.1622(2) & 0.2166(3)\\
Exp. \cite{CCCBDB} & 0.08874(38) & 0.0826(11) & 0.1281(37) & 0.1275(5) & 0.1622(1) & 0.2158(3) \\
\hline\hline
\end{tabular}
\begin{tablenotes}
\item[a] For the smaller systems (LiH, BeH and BH), ECG studies provide the best reference energies. For CH, OH and HF, we combined the atomic energies in Ref.~\cite{Davidson_Atoms} with the atomization energies in \cite{Feller_Corrections} to produce the reference energies.
\item[b] The non-relativistic atomization energy in the adiabatic limit are calculated by subtracting the scalar relativistic, spin-orbit coupling and zero-point energy corrections from the reference energies in Table VI of Ref.~\cite{Feller_Corrections}.
\item[c] Here only the scalar relativistic and spin-orbit coupling corrections are subtracted.
\end{tablenotes}
\end{threeparttable}
\end{table*}

\subsection{Atomization Energies}

\begin{figure}
\centering
\includegraphics[scale=.4]{Figures/atomization}
\caption{Atomization energies of first row hydrides obtained with FN-DMC relative to experimental data. The adiabatic results are estimated by adding zero-point energies from \cite{Feller_Corrections} to the clamped nuclei energies as exact corrections.}
\end{figure}

The ground state energies for various hydrides are reported in Table \ref{tab:atomization}. The energies calculated for clamped nuclei are on par with the best available quantum chemistry results \cite{Adamowicz_LiH,Koput_BeH,Miliordos_BH}. The energies calculated without the adiabatic assumption are in agreement with the ECG results for LiH, BeH to within 0.4 mHa. For small systems, ECG results are typically orders of magnitude more accurate than the best QMC and quantum chemistry simulations. However, with BH being one of the largest ECG simulations performed, the QMC results are actually lower in energy, in this case by 1 mHa. For the systems CH, OH, and HF, there are no explicit simulations we can compare against, and we rely on the experimental results and the results with estimated non-adiabatic corrections for comparison. Our biggest errors appear to occur for BeH and CH. For the case of BeH we agree with accuracy higher than 1 mHa with both the ECG results~\cite{Bubin_BeH_noBO} and semi-empirical benchmark~\cite{Feller_Corrections,Davidson_Atoms}. In particular the ECG results are converged to more digits than the experimental error bar, and it is likely the experimental reference has errors on the order of 5 mHa. For the case of CH, our error is on the order of 3 mHa, which could be the result of the dragged-node approximation.

\section{Conclusion}
We calculated the ground-state energies of first row atoms and their corresponding ions and hydrides to an accuracy of $0.1$ mHa for all but the largest systems both with and without the adiabatic assumption. We found the ionization energies of most of the atoms to be independent of the adiabatic approximation, suggesting that the energy difference between the adiabatic and non-adiabatic ground states is mostly due to the zero point motion of the nuclei. The atomization energies of simple hydrides, however, were significantly different in the adiabatic than in the non-adiabatic limit. We showed that it is necessary to include non-adiabatic effects to accurately predict the experimental values of atomization energies for these simple hydrides to an accuracy of 1 mHa.

These calculations also verified the validity of our wave function ansatz, namely it does indeed produce a high quality electron-ion trial wavefunction from a good electron wavefunction. This technique also has the potential to solve interesting larger-scale problems due to its ease of implementation as well as the polynomial scaling in computational time with respect to the number of electrons.  This technique can be generalized quite easily to deal with larger systems.

\section{Acknowledgment}
The authors would like to thank Mike Pak, Kurt Brorsen, Katharina Doblhoff-Dier and Brian Busemeyer for useful discussions. The authors would also like to thank Prof. Wim Klopper for providing the DBOC references for the atoms and ions and Prof. David Feller for providing the DBOC data for the hydrides. This work was supported by the U.S. Department of Energy (DOE) Grant No. DE-FG02-12ER46875 as part of the Scientific Discovery through Advanced Computing (SciDAC) program. NT and DC were supported by DOE DE-NA0001789. S.H.-S. acknowledges support by the National Science Foundation under CHE-13-61293. J.T.K. was supported through Predictive Theory and Modeling for Materials and Chemical Science program by the Office of Basic Energy Sciences (BES), Department of Energy. We used the Extreme Science and Engineering Discovery Environment (XSEDE), which is supported by the National Science Foundation Grant No. OCI-1053575 and resources of the Oak Ridge Leadership Computing Facility (OLCF) at the Oak Ridge National Laboratory, which is supported by the Office of Science of the U.S. Department of Energy under Contract No. DE-AC05-00OR22725.

\bibliography{ref}
\end{document}
